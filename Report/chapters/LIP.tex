\section{3D-LIP Model with Heading Angles}\label{sec:lip}
If the full dynamic model of the humanoid is used to simulate its motion, it becomes computationally impossible to perform joint path and gait planning, due to its high dimensionality and non-linearity. Therefore, a simplifying model must be used. For this scope Peng et al. introduced the ``3D-LIP Model with Heading Angle'', which describes the discrete dynamics of the Center of Mass (CoM) similarly to the one of an inverted pendulum in three dimensions.

\subsection{Model Definition}
The state $\mathbf{x}$ and the input $\mathbf{u}$ of the dynamic model are defined as:
$$ \mathbf{x} := \left( p_x,\; v_x,\; p_y,\; v_y,\; \theta \right)^T \in \mathcal{X} \subset \mathbb{R}^5 , $$
$$ \mathbf{u} := \left( f_x,\; f_y,\; \omega \right)^T \in \mathcal{U} \subset \mathbb{R}^3 , $$

where $(p_x, v_x)$ are the CoM position and translational velocity along the $x$-axis, $f_x$ is the $x$-coordinate of the stance foot position, $\theta$ and $\omega$ are the humanoid's orientation and turning rate, respectively. $\mathcal{X}$ is the set of the allowed states, while $\mathcal{U}$ is the set of the admissible inputs.

The Linear Inverted Pendulum formulation assumes that, the height $H$ of the CoM is constant during the motion.
According to \cite{peng_main_paper}, we can express the CoM acceleration as:
\begin{align}
    \begin{cases}
        \dot{v}_{x} = \dfrac{g}{H}(p_{x} - f_{x})
        \\[1ex]
        \dot{v}_{y} = \dfrac{g}{H}(p_{y} - f_{y})
    \end{cases},
\end{align}

with $g$ as the magnitude of the gravitational acceleration.
Based on this expression and the previously defined state $\mathbf{x}$ and input $\mathbf{u}$, we can define the 3D-LIP dynamic model in the continuous time space as:

\begin{equation} \notag
\mathbf{\dot{x}}(t) = 
\begin{pmatrix}
    \dot{p}_{x}\\
    \dot{v}_{x}\\
    \dot{p}_{y}\\
    \dot{v}_{y}\\
    \dot{\theta}
\end{pmatrix} =
\mathbf{A_C} \, \mathbf{x}(t) + \mathbf{B_C} \, \mathbf{u}(t) =
\begin{pmatrix}
    \mathbf{A_h} & \mathbf{0}_{2 \times 2} & \mathbf{0}_{2 \times 1} \\
    \mathbf{0}_{2 \times 2} & \mathbf{A_h} & \mathbf{0}_{2 \times 1} \\
    \mathbf{0}_{1 \times 2} & \mathbf{0}_{1 \times 2} & 0 \\
\end{pmatrix} \, \mathbf{x}(t) +
\begin{pmatrix}
    \mathbf{B_h} & \mathbf{0}_{2 \times 1} & \mathbf{0}_{2 \times 1} \\
    \mathbf{0}_{2 \times 1} & \mathbf{B_h} & \mathbf{0}_{2 \times 1} \\
    0 & 0 & 1\\
\end{pmatrix} \, \mathbf{u}(t),
\end{equation}
\begin{equation} \notag
\text{with:}\quad
\mathbf{A_h} \coloneq
\begin{pmatrix}
    0 & 1 \\
    \beta^2 & 0
\end{pmatrix},
\quad
\mathbf{B_h} \coloneq
\begin{pmatrix}
    0 \\
    -\beta^2
\end{pmatrix}, \quad
\beta \coloneqq \sqrt{\frac{g}{H}}.
\end{equation}

By further assuming that the duration of a single humanoid's step is fixed and equal to $T$, we can derive the closed-form step-to-step discrete dynamics of the 3D-LIP model:

\begin{equation}\label{eq:lip_dyanmics}
\mathbf{x}_{k+1} = \mathbf{A_L} \, \mathbf{x}_k + \mathbf{B_L} \, \mathbf{u}_k,
\end{equation}
with:
$$
\mathbf{A_L} \coloneqq 
\begin{pmatrix}
\mathbf{A_d} & \mathbf{0} & 0 \\
\mathbf{0} & \mathbf{A_d} & 0 \\
\mathbf{0} & \mathbf{0} & 1 \\
\end{pmatrix}, \qquad
\mathbf{B_L} \coloneqq 
\begin{pmatrix}
\mathbf{B_d} & \mathbf{0} & 0 \\
\mathbf{0} & \mathbf{B_d} & 0 \\
0 & 0 & T
\end{pmatrix},
$$
$$
\mathbf{A_d} \coloneqq 
\begin{pmatrix}
\cosh(\beta \, T) & \frac{\sinh(\beta \, T)}{\beta}  \\
\beta \, \sinh(\beta \, T) & \cosh(\beta \, T)
\end{pmatrix}, \qquad
\mathbf{B_d} \coloneqq 
\begin{pmatrix}
1 - \cosh(\beta \, T) \\
-\beta \, \sinh(\beta \, T)
\end{pmatrix}.
$$
\\
In the following chapter, these discrete dynamics will be used as the internal process representation of an MPC. Along with the constraints, they will grant that the found solution is meaningful to the humanoid motion.
